
In this section, we will present how we implemented everything. The presentation will be done class by class. All the functions that are prefixed by an underscore should never be used by the user (for example \_popNext). These functions are auxilliary functions that are used by the simulation to generate the data. This convention follows the PEP8 convention of python, as python does not give a way to protect methods.

\subsection{Sensors}

The main class that need to be understood first is the class of the sensors. The idea is to have sensors defined by different manners that will behave the same way. For example, from the simulation class, it will be impossible to know if a data comes from data imported from a file, or from a Markov Chain. The Sensor class is an abstract class that will define the functions that need to be implemented in order to get this behaviour.

\subsubsection{The Sensor abstract class}

The main implementation choice that was made was the one to know how we would generate the data, and send it to the simulation. This is done using a method \verb!_popNext! that will return the next data generated by the sensor. The associated function is \verb!_getNext! that will return the next data that will be generated, but this data is not meant to be used right now (the data can be used to know beforehand when will be the next data generated).

Above that, a sensor have a \verb!name! and a \verb!speed! attributes, with the associated setters (\verb!setName! and \verb!setSpeed!). If the name is not defined by the user, it will be generated automatically as ``Sensor\_i'' where $i$ will be incremented each time a new name is defined (to prevent from two sensors with the same name).
\noteMathieu{Is the speed still useful ?}

Now we should look deeper into the different classes that will inherit from the Sensor class

\subsubsection{The Importer abstract class}

The importer class is also an abstract class that implements the function to read text from an input file. The user have to define a filename at the creation of the class. The real importation of the data (depending on the type of storage of the data) is done in inherited classes.

\paragraph{The JsonImporter class}


