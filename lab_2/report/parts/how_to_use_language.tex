\subsection{Creating the model}

First, the user needs to create the set of sensors that are in their model.
Each sensor has type $\texttt{Sensor()}$. This class is an abstract class.
The user must use sub-classes that define different behaviours.

Each Sensor have a given name. If the name is not defined when building the
sensor, a pre-generated unique name will be used.

There are three kinds of $\texttt{Sensor(name)}$.

\subsubsection{$\texttt{Importer(name)}$}
They are used to get data already generated and stored in different formats
\begin{itemize}
    \item $\texttt{JSONImporter(name, filename)}$ : Reads the JSON file named
        $filename$ and outputs the data contained in it. It must contain the
        following fields:
        \begin{itemize}
            \item "bn" : Name of the sensor
            \item "e" : list of data : each piece of data contains two fields:
            \item "t" : Timestamp when the data was generated
            \item "v" : Value of the sensor at time "t"
        \end{itemize}
    \item $\texttt{csvImporter(name, filename)}$ : Reads the CSV file named
        $filename$ and outputs the data contained in it. The first column must
        be the timestamp. The rest of the columns will be added in the custom
        fields of the \emph{Data} object.
\end{itemize}

\subsubsection{$\texttt{ModelSensor(name, step)}$}
Generates data following a given behaviour. Data is generated on-the-fly. A
value is generated after each $step$ amount of time.

There are some pre-defined $\texttt{ModelSensor()}$.

\begin{itemize}
    \item $\texttt{FunctionSensor(name, step, function)}$ : $function$ is a
python function that will be called on each time step. This allows to use Python
functions, and external modules (for example, call the $np.sin$ function from
the numpy module).
    \item $\texttt{PolynomialSensor(name, step, polynomial, coefs, points)}$ : a
special case of $\texttt{FunctionSensor}$. It implements a polynomial function.
There are three ways to define a polynomial sensor :
    \begin{itemize}
        \item $polynomial$: give the polynomial function
        \item $coefs$: give the array of the coefficients of the polynomial
        \item $points$: give a list of $(x, y)$ values, interpolates the
        polynomial fitting these points
    \end{itemize}
    \item $\texttt{MarkovChain()}$: A Markov chain : use the $\texttt{addState}$,
    $\texttt{addTransition}$ and $\texttt{setStartNode}$ to define a custom chain.
    At each time step, it will compute a transition following the given
    probabilities, and send the new state.
\end{itemize}

\subsubsection{Wrappers}
Modify the behaviour of $sensor$ it contains. It is a wrapper that can catch the
returned data and apply operations on it.

There are three kinds of Wrappers for now:
\begin{itemize}
    \item $\texttt{MaskerSensor(name, sensor, start, stop)}$: forwards the data
of $sensor$ only if the timestamp is between $start$ and $stop$. This allows to
"hide" the sensor during some time, or activate it only on a given period.
    \item $\texttt{SpeedSensor(name, sensor, factor)}$: scales the speed of
$sensor$ by factor $factor$. This allows to set a sensor to run twice as fast,
or twice as slow.
    \item $\texttt{AggregatedSensor(name, sensors)}$: encapsulates a set of
$sensors$. It gathers the data from all $sensors$ and sends them, like a funnel.
\end{itemize}

\subsection{User-friendly ways to define sensors}

Some functions are added to help the user build Sensors. Each function returns
the builded sensor, so they can be stacked.
If $s$ and $s'$ are sensors,
\begin{itemize}
    \item $\texttt{s.named(name)}$ : changes the name of the sensor
    \item $\texttt{s.turnedOnAt(time)}$: disables the sensor before $time$
    \item $\texttt{s.turnedOffAt(time)}$: disables the sensor after $time$.
$time$ can be an absolute time, or a relative time (relative to the start time
of the sensor)
    \item $\texttt{s.turnedOnBetween(start, stop))}$: both at the same time
    \item $s + s'$: builds a sensor sending the data of both sensors
    \item $s * n$ with $n$ an integer : builds a sensor containing $n$ copies
of $s$
\end{itemize}

So, if $s$ models the sensor of a parking slot, $(s*30).named("parking")$ is a
sensor containing 30 independent parking slots, and returning the data of each
slot. This multiple sensor is called "parking"

\subsection{Preparing the display devices}
Displays the simulated data to a format where it will be visualised.

\begin{itemize}
    \item $\texttt{InfluxDBDisplay(url)}$ : outputs the data to an influxDB data
base running at url $url$
    \item $\texttt{JSONDisplay(filename)}$ : outputs the data in a JSON file
called $filename$
\end{itemize}


\subsection{Creating the simulation}
The $\texttt{Simulation}$ object is built with all the previous objects:
$\texttt{Simulation(name, display, sensors, speed, start, stop, realtime)}$

\begin{itemize}
    \item $name$ : name of the simulation
    \item $display$ : the display where to export the data
    \item $sensors$ : list of the sensors to simulate
    \item $speed$ : speed factor to speed up or slow down the whole simulation
    \item $start$ : timestamp when the simulation starts
    \item $stop$ : timestamp when the simulation ends
    \item $realtime$ : if $True$, the simulation will wait and make sure the
data is generated at the good timestamp, nothing will be pre computed.
\end{itemize}

Once the simulation $simu$ is defined, a simple $simu.run()$ will start the
simulation.

\subsection{Examples}
The easiest way to discover how to build the simulation is to look at the
examples given in the folder, they depict most of the functionnalities.
