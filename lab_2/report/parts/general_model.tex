

The general model is divided into three main components:
\begin{itemize}
\item Simulation
\item Display
\item Sensor
\end{itemize}

Then there is also two components that are more general classes: \emph{Timestamp} and \emph{Data}.

\subsection{Simulation}

The simulation is the core class that will run the simulation. In this class, the user should be able to define the speed at with he wants to run the simulation (for example run a simulation twice as fast as it should be). He will also define the sensors that will be used, and the display where whe wants to output the data. Finally, the method run will launch the simulation.


\subsection{Display}

This class will deal with the output of the simulation. Many different outputs can be defined: CSV, JSON formats, or outputing in an influxDB database (that can be seen with Grafana).


\subsection{Sensor}

This is the most interesting class that will allow the user to define sensors. The sensors here will be defined according to different models, and then they should work all the same way, no matter what is their definition (for example a Markov chain sensor should look the same as a sensor mimicing the data coming from an existing database.


\subsection{Timestamp}

TODO


\subsection{Data}







